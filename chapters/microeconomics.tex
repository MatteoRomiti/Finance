\chapter{MICROECONOMICS - 14.01}

\section{Market inefficiency}
If prices for a good increase, we are not immediately able to tell if it was due to a shift in the demand or in supply. 

In the labor market, the firms are the demand, the workers are the suppliers. The supply curve with hours worked per year on the x-axis and wage on the y-axis is an upward-sloping, because the higher the wage, the more you're willing to work. The demand is instead downward-sloping: the higher the labor cost, the lower the number of people hired (machines would maybe replace them).

Right-wing people would support markets and let them free to evolve as they always tend to equilibrium. Left-wing people is instead prone to government intervention. 

An example is the introduction of minimum wage. If this minimum wage is above the current equilibrium, firms will be willing to hire less workforce at the minimum wage and workers will be willing to work more for the minimum wage. This gap is unemployment. If the minimum wage is below the equilibrium it would be ignored.

Another example is the following. Imagine we have problems in oil production and the supply curve moves up, but the government decided to put a cap on the price of oil to bring it back to the previous price. This would lower the price but would also create an excess in demand. At this lower price, oil producers will be willing to produce less, while consumers to buy more.

Efficiency in markets is when every trade which makes both parties better off is made, without interference from other parties. When this does not happen, markets become inefficient and losses occur. In the two examples above we have inefficiency in that unemployed people might be willing to work for a wage lower than the minimum and firms might be willing to hire people for a lower wage, but this cannot happen. Same point for the oil price cap: People that do not get enough oil are willing to pay more to have it, instead of waiting in line and waste time, and oil producers are willing to provide more oil for a higher price, but this cannot happen. Allocation inefficiency occur. The benefit though is that with these measures, there is more equity in the sense of fairness: everybody can get oil regardless of how rich they are, everybody can compete to have a minimum wage, and nobody is underpaid (despite some are not paid at all).

One solution to the problem of oil shortages (or also water shortages) would be to create secondary markets where people gets a minimum amount of the resource, decided by the government, and they can trade it depending on their needs.

The efficiency in the market would also mean that Americans and Europeans should ship their garbage to Africa and pay for it because we have money and garbage, they have land to store garbage and no money. Both parties would be better of.

\section{Elasticity in demand}

When a product has a perfect substitute (imagine two very similar kinds of biscuits for example), then the price of the product is perfectly elastic, because one can immediately change it with its substitute. In the other extreme, a product with no substitute (imagine a specific medicine) is perfectly inelastic in price to shifts in supply. One would always buy the product, regardless of its price. The percentage change in quantity over the percentage change in price is the elasticity of a product. The two extreme scenarios presented above correspond to negative infinite elasticity and zero elasticity.

\begin{equation}
    \epsilon = \dfrac{\Delta Q/ Q_i}{\Delta P/ P_i}
\end{equation}
Where $P_i$ is the initial price before the shift.
The change in revenue with respect to the change in price is:
\begin{equation}
    \Delta R / \Delta P = Q (1 + \epsilon).
\end{equation}
Hence, when elasticity is greater than -1, it makes sense to increase prices. To determine elasticity one need empirical economics and not theoretical economics. In particular, we need to know the multiplier of the demand curve (not the multiplier of the supply curve otherwise we confuse causation with correlation).

What if there is a new tax on a specific product and this product is completely elastic? The producer will try first to increase the price, to cover the taxes, so the supply curve shifts up, but the equilibrium will be at the point where the price is the same as before and lower production. When the government wants to raise money through taxes, it should do so taxing inelastic products. Liberals might want to raise taxes on yachts, but they are very elastic, while cigarettes are not. When the politicians or journalists say the government will raise a certain amount of money by taxing a specific good, they are just reporting the best estimates. Also, elasticity cannot be constant with big supply or demand changes.


\section{Stock options to managers}

if a manager is given stocks of his company, he will take decisions more wisely and avoid risky moves. If he gets stock options instead he will take risky decisions because he will not get the downsize of a bet. If the company loses value, he is covered by the (call) option. Investigation done by the WSJ revealed that many managers were given backdated stock options, meaning that they were artificially issued the day before the company's value increased and the manager was given free money in an apparently-legal way. This is of course fraud, but the managers don't care because if they get caught, they'll pay a fine for a third of what they made.

% monopsony is the power of a company over its labour. People prefer to work with the company at a lower wage despite the fact that there are other companies paying more. This is because the worker can't leave the city of the company or does not have other skills for change company. In this case a well-set minimum wage can reduce unemployment, just like a monopoly with a regulated price can improve welfare. How can this be possible? Maybe companies can afford to pay more and still be able to operate. Higher wages means more consumption and more demand and more supply and higher wages.

When we say we want to redistribute money from richer to poorer people, we can not simply take money from richer people and put it in a bucket and give this bucket to poor people, because there is two sources of leakage. First, we increase taxes on rich people and they will start to work less hard and production decreases and the pie shrinks. Second, almost poor people will start quit their jobs to qualify as poor and this also shrinks the pie. Social waste, deadweight loss arises from this redistribution system. Is it worth it? You need to solve the social welfare function, which we have to choose. Equity-efficiency trade off.
In the US, for every dollar they try to put in the bucket, roughly 40 cents leak out.
To reduce this leakage, targeted redistribution of income was implemented: poor disable people and poor single-parent family receive money.

Income is consumption plus savings. In the US, income is taxed more than in Europe, where consumption is taxed more instead. In Europe, hence, savings are not as taxed as in the US and they are promoted. But consumption taxation is very regressive, it hits the poor more than the rich. Governments in Europe though make sure to take care of poor people.

The United States federal earned income tax credit or earned income credit (EITC or EIC) is a refundable tax credit for low- to moderate-income working individuals and couples, particularly those with children. The amount of EITC benefit depends on a recipient's income and number of children. It works as a patch to the bucket.

In the US, \$80B a year are associated to costs for smoking. Smoking, obesity, alcohol, gasoline.

to solve the asymmetric information in insurance the government can do two things: subsidize or mandate. The first, the government pays the insurance for everyone, but has to raise taxes and everyone is affected. The second, the government by law says everyone must buy an insurance. In this case people not in need of an insurance will be upset. In the first case we have inefficiency in raising taxes and taxpayers will be upset. There is no correct answer in this market failure.