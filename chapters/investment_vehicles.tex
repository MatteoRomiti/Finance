\chapter{INVESTMENT VEHICLES, INSURANCE, AND RETIREMENT}
\begin{chapquote}{George Bernard Shaw}
``The reasonable man adapts himself to the world; the unreasonable one persists in trying to adapt the world to himself. Therefore, all progress depends on the unreasonable man.''
\end{chapquote}

\section{Mutual funds and ETFs}
Imagine there is a good investor with a lot of experience in finance that want to start a company. He goes to the SEC (Securities and Exchange Commission) to register his corporation (management company) and himself. Then he set up his business with come cash as asset and issues some stocks to himself. In an open-end mutual fund, anyone can ask him for stocks and become owner of the company, but the founder will keep an additional percentage of the assets for managing the business. In addition, anyone can decide to leave and cash out the stocks, so the manager must always have some liquid asset. AUM stands for assets under management. NAV stands for net asset value. The value of a share changes only thanks to good or bad investments of the manager. When shares are issued or redeemed (which happens at the end of the day), there is no change in stocks value. 

In a closed-end fund, the manager starts the business with some shares and some initial investors. After that, there is no change in number of stocks: no issuing, nor redemption. This means the manager does not need to keep cash and can invest all the assets. If a shareholder wants to cash out, she has to find an investor and sell the share in a secondary market (NASDAQ, NYSE).

To get the best of both worlds, there are ETFs (exchange traded funds). They are flexible as an open-end mutual fund, but usually only big institutions buy from or sell to the fund manager (lots of shares) and in addition these big investors can sell shares in exchange markets to the public. Usually big ETFs are not actively managed by an expert investor, rather they buy some commodities or some asset class (S\&P500, gold) so there are lower management fees.

\section{Hedge funds}
Hedge funds differ from mutual funds in that they are not regulated by the SEC therefore they can't market themselves (no ads) and can't take money from the public, only from accredited investors (a certain net income or a certain education). A hedge fund is more actively managed and has higher fees, but also the manager takes a certain percentage of the profit (around 20\% for taking the risk) called performance fee or carried interest.
In a hedge fund, the manager can have shares (called limited partner interests) of the fund, but not under his name, rather he uses a LLC. When there is a profit, roughly 2\% is the management fee, and after deducting this amount, the fund manager takes another 20\% roughly. An investor can redeem or invest money in the fund at certain points in time, so it is an open-end fund from this point of view.
The limited partners (shareholders) make sure that the general partner (fund manager) is a reliable investor. The activity of the general manager is usually not fully disclosed to the partners as it may help other traders. This secrecy is sometimes hiding some shady activity. In general, though, a hedge fund becomes dangerous for society when it is too big to fail. If it keeps a smaller size, the only people in danger are the partners. 

Venture capital and private equities are similar to hedge funds in that they have a management fee and a profit percentage, but they invest on small businesses where there is not much liquidity upfront. Hedge funds can turn assets into cash faster and easier.

Let's say I predict that the market will go up, but I'm confident one company, B, will grow more than the other, A. Then, I short a sum $S$ on company A and I long the same amount on company B. If both companies grow, but B grows more than A, I make profit, otherwise I lose money if A grows more and my prediction was wrong. If instead I predict the markets will go down but again company A will do worse than B, I make profit. Long/short hedge funds are doing this: they pick two companies and predict which one will do better (or less worse) compared to the other, regardless of the market going up or down.

When there is an acquisition incoming, where company B acquires company A paying $v'$ for each share of A, more than the current value $v$, A's shares will be traded around $V + pV$, where $p$ is a sort of market measure of the probability the acquisition will happen. If you think the acquisition will happen, you'll buy A's shares, otherwise you short, because they'll lose money. This is a merger arbitrage, a strategy of hedge funds~\footnote{"Hedge funds always find out first". GoBlue on Khan Academy's lesson}.

\section{Retirement accounts: IRAs and 401ks}
An individual retirement account is a deposit where a person puts a sum $S$ of money every year until a certain age. This money is tax deductible and can be cashed out paying a penalty and taxes on it (no penalty if after a certain age and probably there will be a lower tax bracket). This sum can be used for investments and make profit without capital gain taxes. Withdrawal is mandatory after a certain age.

With a Roth IRA (named after William Roth), you pay taxes when you deposit money %every time?%
and you won't be taxed when you cash out, even if you made big profit with your IRA. If you want to withdraw your money before a certain age, you don't pay any penalty or taxes on principal (10\% if more than the principal). A traditional IRA can be transferred to a Roth IRA. Withdrawal is not mandatory.

A 401(k) is similar to a traditional IRA but has some differences: higher % annual?
limit of money you can put in, organized by your employer, meaning that you don't manage the investments), the employer might match it (add some money), you can borrow from your 401(k) with some interest.

\section{Life insurance}
There are two possible ways to ensure life: paying a certain amount of money every year for the rest of my life or set a term so that I'll be covered for a specific period and then I won't pay anymore. Let's assume I pay $c$ per year over the next $N$ years for a life insurance of $I$. At the end of the term I would have paid $c\cdot N$, which means the company gets $c\cdot N$ on every insurance of $I$. To break even, the company must give the same insurance to $\dfrac{I}{N \cdot c}$ people and hope that at most one person dies among them during the term.

\section{Investment and consumption}
In ancient times, the word capital was used to describe any object used to produce a good such as tools, equipment and buildings. Human capital instead includes everything that make people more productive such as education, experience, talent. Financial capital is money used for the purpose of producing more output.

One possible definition of return on capital is the ratio between the amount of money made in one year and the money you invest in the business $ROC = \dfrac{S'_y}{S}$. To understand whether you should start a business you also have to take into account the costs of starting the business (interest rate of a loan). Different businesses may have different interest rates, be it for government incentives or taxation.

We can say that an investment is when money is spent to improve society. An example can be creating a factory producing cheaper cars or a family buying a house allowing their members to have a decent life. The tricky part is: what if I add a bathroom in my house? Investment or consumption? If that bathroom allows the tenants of the house to get ready in the morning faster to go to work for example, then it should be considered an investment. But if that bathroom is used only by guests and seldom, then it is considered consumption. In addition, if you consider this latter case as an investment, meaning that you think you'll find another person willing to buy the house at a higher price because of the bathroom, you are instead speculating. In other words, consumption is enjoyment, but also the loss of the opportunity to use that same money to have more enjoyment in the future. To be more precise, one can spend money for an investment and for consumption at the same time. An example is buying a purse when you really need it to be more organized at work. You can buy a simple one where 100\% of your money is invested or you buy an expensive one where, say, 20\% is investment and the rest is consumption or transfer of wealth. The money spent on advertising that purse is consumption instead, there is no improvement in society. Any activity that may waste people's time is considered consumption.

Let's imagine you have a house which you bought for $S$ and it's now yours (no debt). After 10 years, the value of your house increased and it's worth now $S'$. You feel richer, obviously, and you decide to take a home equity loan of $S'/2$ with a certain interest rate $r$ and you consume that sum say on vacation. After one year, your house suddenly loses its value and it's worth now $S''< S$. You, owner of the house, decide to give it to the bank instead of paying the loan $S'/2$ (plus interest). The bank auctions the house and gets $S''$ as planned. So the bank lost money, in particular they lost $S'/2 - S''$. If you didn't take that home equity loan, you could have sold your house now at $S''$. Instead, you lost that money. Now you're left with no house (no equity) but you consumed $S'/2$. The total amount of money lost is then $S'/2-S''+S'' = S'/2$, which is exactly the money consumed on your vacation. That sum could have been used by the bank for financing a company.
