\chapter{SCANDALS}
\begin{chapquote}{Jeb Hensarling}
``How can we have capitalism on the way up and socialism on the way down? If we lose our ability to fail will we not in turn lose our ability to succeed?''
\end{chapquote}

What follows is not part of the Khan Academy's course, rather an analysis of some of the scandals related to finance in recent times. This will help myself and the reader to better understand when and how finance can be harmful for society.

\section{Cartel Bank}
% anabel hernandez - Mexican journalist
% victor avila - special agent
% sinaloa region cartel
% ken del valle - criminal defense attorney
% patrik radden keefe - the new yorker
% matt taibbi - rolling stone
% william ihlenfeld - us attorney
% brett wolf - anti-money laundering correspondent thomson reuters
% Loretta Lynch, US attorney, condemned HSBC to pay a fine 
Imagine a guy in Mexico with a regular job. He has to work for more than six months to get \$3000, but he also has the choice to cross the US border driving a car with 300 pounds of marijuana, stop at a shopping mall for few hours, get back to the car with no more drug in it and come back home. He would make the same amount of money in one afternoon. And this is not the difficult part. The drug will be sold on the street and lots of cash will be collected. The difficult part is then to put these US dollars into the financial system. Or maybe it's not so difficult. 

"Plata o plomo", silver or lead, bribe or death. These are the famous words usually told to a banker to convince him to launder the money. In Mexico, there was a bank called Banco Bital, with a strong business in regions related to drugs. Banco Bital was bought by HSBC in 2002 and it said that Bital was not a risky bank.

HSBC Holdings plc is instead a British multinational banking and financial services holding company. It is the largest bank by total assets in Europe with total assets of US\$2.374 trillion (as of December 2016). HSBC stands for Hongkong Shanghai Banking Corporation.

Zehnli Ye Gon, a Mexican also known as el Chino, was running Unimed, a pharmaceutical business and it was used to bring meth into the US. In his mansion, guns and stacks of cash (205 million USD) were found. He was a customer of HSBC Mexico. HSBC London ordered to close his bank account in 2004, but it was not closed.

Barton Joseph Adams, a doctor in West Virginia, was running Interventional Pain Management. He prescribed opioid to his patients, committing fraud with Medicare and West Virginia Medicaid. He underreported his income coming from this scheme and laundered money. The transactions were with HSBC. Adams was just the tip of the iceberg. Many of the other transactions found were related to terror financing, Russian criminals and more.

In light of this scandal HSBC was forced to assume people and clear these transactions. Everett Stern was one of those, but he was not a criminal, nor a stupid. He was asked to go through the system and close the alarms related to these transactions. He discovered that HSBC allowed transactions from criminal organizations, such as Tajco, a financial supporter of Hizballah, one of the most dangerous terrorist groups in the world. The OFAC sanctions list contains details about these entities and Tajco is one of them. 

Stuart Levey, Chief Legal Officer of HSBC and treasury official of the United States, and Irene Dorner  CEO of HSBC Bank, admitted the mistakes many times, but did not solve the problems. The discussion between Senator Carl Levin and the HSBC bosses looks very similar to the one Mark Zuckerberg had with the senators in light of the Cambridge Analytica scandal.

William Ihlenfeld, an attorney in West Virginia was asked to step down when he reported the problems with Doctor Adams. He was said to stand down from looking at HSBC.
HSBC paid a fine of \$1.9 million, five week's profit and a deferred prosecution agreement (DPA) was made. A deferred rosecution agreement is very similar to a non-prosecution agreement (NPA). It is a voluntary alternative to adjudication in which a prosecutor agrees to grant amnesty in exchange for the defendant agreeing to fulfill certain requirements. In this case, these requirements were to partially defer bonus compensation for its most senior executives. Wow! A journalist during the conference raised the point that the fine HSBC paid was small compared to what they did. The answer from Lanny A. Breuer, an American lawyer was: "Our goal here is not to bring HSBC down, it's not to cause a systemic effect on the economy, it's not for people to lose thousands of jobs", and the journalist "Don't they [the senior executives probably] deserve that?". No individual was fined, no one went to jail.

Eric Holder, Attorney General of the United States from 2009 to 2015, was asked not to indict HSBC Bank and he agreed. It happened when the US was coming out of the 2007 financial crisis and another hit to a bank would have raised alarms again.

Elizabeth Warren: "If you're caught with an ounce of cocaine, you'll likely go to jail. If it happens repeatedly, you may go to jail for the rest of your life. But evidently, if you launder nearly a billion of dollars for drug cartels and violate our international sanctions, your company pays a fine and you go home and sleep in your own bed at night. Every single individual associated with this. I think it's fundamentally wrong. That is not equal justice under law".

In 2012, the same year HSBC avoided prosecution, over 90,000 people were sentenced to federal prison for drug offenses in the USA. Since 2002, over 100,000 people have been killed in Mexico as a result of drug cartel violence. From 2002 to 2014, heroin-related overdose deaths in the USA more than quadrupled.

From the official document \citep{HSBC}:

"From 2006 to 2010, HSBC Bank USA violated the BSA [Bank Secrecy Act of 1970] and its implementing regulations. Specifically, HSBC Bank USA ignored the money laundering risks associated with doing business with certain Mexican customers and failed to implement a BSA/AML [Anti Money Laundering] program that was adequate to monitor suspicious transactions from Mexico.

As a result of these concurrent AML failures, at least \$881 million in drug trafficking proceeds, including proceeds of drug trafficking by the Sinaloa Cartel in Mexico and the Norte del Valle Cartel in Colombia, were laundered through HSBC Bank USA without being detected. HSBC Group was aware of the significant AML compliance problems at HSBC Mexico, yet did not inform HSBC Bank USA of these problems and their potential impact on HSBC Bank USA’s AML program.

There were at least four significant failures in HSBC Bank USA’s AML program that allowed the laundering of drug trafficking proceeds through HSBC Bank USA:
a. Failure to obtain or maintain due diligence or KYC [Know Your Customer] information on HSBC Group Affiliates, including HSBC Mexico; 
b. Failure to adequately monitor over \$200 trillion in wire transfers between 2006 and 2009 from customers located in countries that HSBC Bank USA classified as “standard” or “medium” risk, including over \$670 billion in wire transfers from HSBC Mexico; 
c. Failure to adequately monitor billions of dollars in purchases of physical U.S. dollars ("banknotes") between July 2006 and July 2009 from HSBC Group Affiliates, including over \$9.4 billion from HSBC Mexico; and 
d. Failure to provide adequate staffing and other resources to maintain an effective AML program."

On December 11, 2017, the US Department of Justice announced its plans to dismiss all criminal charges against HSBC

The whole story at \url{https://www.netflix.com/title/80118100}

% \section{Hard NOx}

% % alex gibney - director
% % sally yates - deputy attorney general
% % jack eving - reporter
% % walter groth - ex executive of volkswagen
% % Ferdinand piech - vw and audi executive
% from year 2009 to 2015, 500 000 diesel vehicles 